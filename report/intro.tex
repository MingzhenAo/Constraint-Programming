\chapter{Introduction}
\label{cha:intro}
The goal of this project is to model two puzzle games by building their CSPs and then compare the performances of different solvers to these CSPs.
\section{Introduction}
\label{sec:introduction}
In March 2016, AlphaGo won a one-week-long Go series by 4-1 against Lee Sedol, a worldwide top Go player~\cite{r26}. This amazing result indicates that computers nowadays have huge advantages over humans in some games like Go, which contains finite possible cases. Hence, if a powerful computer works with efficient algorithms, it would significantly boost the speed of solving some problems. Mathematical models, such as CSPs, can be a bridge that connects real-world problems with computers by translating nature language to numerical or logical symbols and statements, through which the computer can understand and then solve real-life problems where complicated calculations couldn’t be solved manually. CSP has been developed since a long time ago when it originated in the field of artificial intelligence in 1978 by Jena-Lonis Lauriere~\cite{r27}. At that time, it was limited to poor computing power so that it could hardly solve any problems with high complexity within a reasonable time. When faced with them, they may even perform worse than human did. However, with more efficient optimization methods developed and more powerful computers manufactured, CSPs has been widely used to solve some complex problems in many aspects such as location, scheduling, car sequencing, vehicle routing, timetabling, rostering~\cite{r28} etc. Meanwhile, along with the wide applications of CSPs, many compilers have supported the constraint satisfaction programming language. And massive researchers in the world have created different solvers that adopt different optimization methods to solve such a problem. Due to the different features of the solvers, the performance of each solver to different problems may be significantly different. Therefore, the comparisons between them are meaningful, meanwhile, it helps the modeler to choose the suitable one.
\section{Thesis Outline}
\label{sec:outline}
In this thesis, some basic background knowledge such as CSP, rotation matrix and Minizinc language are introduced in Chapter~\ref{cha:background}. The details of CSP models and corresponding encodings in Minizinc for both IQ Twist and Zig Zag Puzzler are presented in Chapter~\ref{cha:design}. In Chapter~\ref{cha:evaluation}, there is a discussion about the results of nine different solvers. In the end, the conclusions are drawn and prospects on future work are described in Chapter~\ref{cha:conc}.
