\chapter{Introduction}
\label{cha:intro}
\section{Thesis Statement}
\label{sec:problemstatement}
The goal of this project is to model two puzzle games based on CSPs and compare the performances of different solvers.
\section{Introduction}
\label{sec:introduction}
In March 2016, AlphaGo played Go with Lee Sedol who was one of the best Go players in the world, after five games in a week, AlphaGo won the series 4:1~\cite{r26}. A computer program defeats the best player in the world, which indicates that the computer takes huge advantages compared with the human in such a game that contains a finite of possible configurations. Hence, the effective use of the computer can significantly accelerate the speed of solving some problems. A mathematical model like a CSP can be a bridge that connects the real-world problem with the computer. Through the bridge, the computer can help human solve problems in real life. CSP has been developed in a long history, it originated in the field of artificial intelligence in 1978 by Jena-LonisLauriere~\cite{r27}. At that time, it was limited to insufficient computing power so that it is hard to solve the problems with high complexity within a reasonable time. The performance may be even worse than the performance of human solving it. However, with the development of optimization methods and the growth of computing power, it has been widely used to solve some complex problems in many aspects such as location, scheduling, car sequencing, vehicle routing, timetabling, rostering and so on~\cite{r28}. Meanwhile, along with the wide applications of CSPs, many compilers have supported the constraint satisfaction programming language. And massive researchers in the world have created different solvers that adopt different optimization methods to solve such a problem. Due to the different features of the solvers, the performance of each solver to different problems may be significantly different. Therefore, the comparisons between them are meaningful, meanwhile, it helps the modeler to choose the suitable one.
\section{Thesis Outline}
\label{sec:outline}
In this thesis, some basic background knowledge such as CSP, rotation matrix and Minizinc language are introduced in Chapter~\ref{cha:background}. The details of CSP models and corresponding encodings in Minizinc for both IQ Twist and Zig Zag Puzzler are presented in Chapter~\ref{cha:design}. In Chapter~\ref{cha:evaluation}, there is a discussion about the experiment results of nine different solvers. Moreover, the conclusion and future work is described in Chapter~\ref{cha:conc}.
