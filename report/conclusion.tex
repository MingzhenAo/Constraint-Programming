\chapter{Conclusion}
\label{cha:conc}
In conclusion, based on the CSP models of IQ Twist (the 2D game) and Zig Zag Puzzler (the 3D game). There are 120 problems created for IQ Twist, and 80 problems created for Zig Zag Puzzler through Minizinc. All of these problems come from the game booklets. After the creations of these problems, nine solvers that are based on different languages are used to evaluate each problem. Finally, the coverage rates of each solver and the execution times for each problem by different solvers have been discussed, which indicates that the Chuffed, PicatSAT and OR-Tools can cover all the problems, meanwhile, they need shorter execution time for solving most problems. In addition, Chuffed achieve the optimal performance because even for the highest difficulty, it solves each problem in 20 seconds for both games. In comparison, the other two solvers usually need more execution time.
\section{Future Work}
\label{sec:future}
In the future, I would like to figure out why Chuffed, PicatSAT and OR-Tools can cover all the problems in a short time. Some of their features have been mentioned in Chapter~\ref{section:compiler}. 
For Chuffed, the reasons why lazy clause generation and nogood logbook can achieve such an optimal performance might be interesting. 
For OR-Tools, what the combination optimization methods of OR-Tools are and why they can achieve good performance are worth considering.
For PicatSAT, the Picat language and the advantages of log encoding may be worth learning.
Of course, except for the features that I mentioned, there might be other special parts of each solver. After figures out their advantages, I may try to create another solver that combines their advantages.



